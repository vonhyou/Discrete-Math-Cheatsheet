% This template is made by Shuo Feng, for academic use.

\documentclass[]{article}
\usepackage[utf8]{inputenc}
\usepackage{amsfonts}
\usepackage[letterpaper, total={7in, 9in}]{geometry}

% For multi cols
\usepackage{multicol}
% For set signs
\usepackage{mathtools}

% Use LaTeX font EB Garamond
% https://tug.org/FontCatalogue/ebgaramond/
%\usepackage[cmintegrals,cmbraces]{newtxmath}
%\usepackage{ebgaramond-maths}
%\usepackage[T1]{fontenc}
% Font usage end
  
\title{\scshape{Discrete Math Cheat Sheet}}
\author{vonhyou}
\date{Apr. 07, 2022}

\begin{document}

\maketitle

\begin{multicols}{2}
[
\section{Function, Sets and Relations}
A \textbf{function} from set $ \mathbb{A} $ to set $ \mathbb{B} $ is a mapping such that every $ a \in \mathbb{A} $ is mapped to a unique $ b \in \mathbb{B} $ . A \textbf{set} is a collection of distinct objects. The \textbf{Cartesian Product} of sets $ \mathbb{A} $ and $ \mathbb{B} $ is the set of all pairs whose first component is from $ \mathbb{A} $ and second component is from $ \mathbb{B} $. We define \textbf{n-ary relation} R as a subset of an n-product, if n = 2 we call R a (binary) relation $ \mathbb{R} \subseteq \mathbb{A}_1 \times \mathbb{A}_2 $. If $ \mathbb{A}_1 = \mathbb{A}_2  = \mathbb{A}$ we call \textbf{R is a relation on A}.
] 

\begin{enumerate}
  \item Examples of defining function
    \begin{flalign} \nonumber
        & f =  \{(n, |n| + 2) : n \in \mathbb{Z} \} && \\\nonumber
        & f \; :  \; \mathbb{Z} \rightarrow \mathbb{N} && \\\nonumber
        & \qquad n \mapsto |n| + 2 
    \end{flalign}
    
  \item Ways to define a function
    \begin{itemize}
      \item Directly (a = 1, b = 2)
      \item Piece wise
      \item Mathematical expressions ( $x \mapsto cos(x)$ )
      \item From existing functions 
    \end{itemize}
    
  \item Domain, Codomain and Range
    \begin{itemize}
      \item Domain/ Range: The input/ output set
      \item Codomain: The set contains the output set
    \end{itemize}
  \item Well-defined functions \\
    The function is defined for \textbf{all inputs}, and every input has a unique output. Use \textit{vertical line test}
  \item Boolean functions
    \begin{flalign} \nonumber
      and: \; & \mathbb{B} \times \mathbb{B} \rightarrow \mathbb{B} && \\\nonumber
              & (0, 0) \mapsto 0, (0, 1) \mapsto 0 && \\\nonumber
              & (1, 0) \mapsto 0, (1, 1) \mapsto 1 
    \end{flalign}
  \item Injective, Surjective and Bijective
    \begin{itemize}
      \item Injective: One-by-one, \textit{horizontal line test}
      \item Surjective: Onto, every output has input(s)
      \item Bijective: One-by-one and onto
    \end{itemize}
  \item Prove surjective: $ g(m, n) = (a, b) = (m + n, 2m + n) $ . We get $ m = b - a $ and $ n = 2a - b $ . For any, so surjective.
  \item Rational(fractions): $ \mathbb{Q} = \{ \frac{a}{b} \; | \; a, b \in \mathbb{Z} \} $
  \item Subset $ \subseteq $ and \textbf{Proper} subset $ \subset $
  \item \textbf{Cardinality}: The number of elements in the set
  \item Union: $ \cup $, Intersection: $ \cap $, Difference \textbackslash
  \item \textbf{Power set}: The set of all subsets
  \item A pair with \textbf{components} a and b is (a, b) 
  \item \textbf{Cartesian Product}: $ \mathbb{A} \times \mathbb{B} $ \\
    $ |\mathbb{A} \times \mathbb{B}| = |\mathbb{A}| \times |\mathbb{B}| $
  \item \textbf{n-tuple} is a pair of n components 
  \item \textbf{n-product} is the Cartesian product of n sets
  \item Relation as a direct graph is a pair of \textbf{vertices} (dots) and edges.
  \item Direct graph, edges have direction. 
  \item Relations
    \begin{itemize}
      \item Reflexive: $ \forall a \in \mathbb{A} \Rightarrow (a, a) \in \mathbb{R} $
      \item Symmetric: $ \forall (a, b) \in \mathbb{R} \Rightarrow (b, a) \in \mathbb{R} $
      \item Transitive: $ (a, b) \; and \; (b, c) \in \mathbb{R} \Rightarrow (a, c) \in \mathbb{R} $
    \end{itemize}
  \item Closures
    \begin{itemize}
        \item Reflexive Closure: $ \mathbb{S} = \mathbb{R} \cup \{(a, a)\; |\; a \in \mathbb{A} \} $
        \item Symmetric Closure: $ \mathbb{S} = \mathbb{R} \cup \mathbb{R}^{-1} $ 
        \item Transitive Closure: Remember to check if newly added edges creates new connected pair.
        \item Equivalence Closure: Reflexive, symmetric and transitive.
    \end{itemize}
  \item \textbf{Vacuous Truth}
\end{enumerate} 

\end{multicols}

\begin{multicols}{2}
[
\section{Logic and Proof}
\textbf{Propositional logic} studies ways to of joining and/or modifying entire propositions, statements or sentences to form more complicated of them.
]

\begin{enumerate}
  \item A \textbf{proposition} or a \textbf{statement} is a sentence that is either true or false but not both.
  \item A \textbf{proposition} is $ P $, and an \textbf{open proposition (sentence)} is $P(x)$. \textbf{Atomic} propositions cannot be simplify. \textbf{Compound} propositions combine atomic propositions using logical operators.
  \item Logical connectives \\
    \begin{tabular}{|l|c|c|}
    \hline
    Negation      & $\neg$             & not                          \\ \hline
    Conjunction   & $\land$            & and                          \\ \hline
    Disjunction   & $\lor$             & or                           \\ \hline
    Exclusive or  & $\oplus$           & xor (but not both happen)    \\ \hline
    Conditional   & $\Rightarrow$      & if then (implies)            \\ \hline
    Biconditional & $\Leftrightarrow$  & iff (if and only if)         \\ \hline
    \end{tabular}
  \item $ P\Rightarrow Q = \neg P \lor Q$
  \item $ P\Rightarrow Q $, P is sufficient for Q. Q is a necessary condition for P. P cannot be true if Q is false.
  \item Concepts related to truth table
    \begin{itemize}
      \item Logically equivalent: $P \Leftrightarrow Q$ is a tautology
      \item Contradiction: Always false
      \item Contingency: Neither tautology nor contradiction
      \item Contrapositive: $P \Rightarrow Q = \neg Q \Rightarrow \neg P$
    \end{itemize}
  \item \textbf{Literal}: Total number of appearances of each variable. $\neg PQR+RS$ has 5 literals.
  \item \textbf{Optimal}: Has fewest number of literals
  \item Rules of Inference
    \begin{itemize}
      \item Modus Ponens: $P \Rightarrow Q$, P is true, so Q is true
      \item Modus Tollens: $P \Rightarrow Q$, Q is false, so P is false
      \item Elimination: $P \lor Q$, P is false, so Q is true
    \end{itemize}
  \item Karnaugh Map: Remember Gray Codes
  \item If we have the implication $P \Rightarrow Q$
    \begin{itemize}
      \item \textbf{Converse}: $Q \Rightarrow P$
      \item \textbf{Contrapositive}: $\neg Q \Rightarrow \neg P$
    \end{itemize}
  \item A \textbf{theorem} is a statement that can be (has been) verified as true. A \textbf{proof} of a theorem shows how.
  \item A \textbf{proposition} is a theorem that may not be as important as another theorems. A \textbf{lemma} is a theorem whose purpose is to prove another theorem. A \textbf{corollary} is a result that is an immediate consequence of a theorem.
  \item Typical proof steps
    \begin{itemize}
      \item Start with \textbf{premise} (assumption)
      \item Build off the premise by deduction or inference
      \item Conclusion
    \end{itemize}
  \item Types of Direct Proofs
    \begin{itemize}
      \item Proof by cases: Categorize discussion
      \item Proof using sets
      \item Proof via example
    \end{itemize}
  \item Types of Indirect Proofs
    \begin{itemize}
        \item \textbf{Contrapositive}: Assume $\neg Q$, if works, $\neg P$
        \item \textbf{Contradiction}: Assume $\neg P$, if works, $Q \land \neg Q$
        \item \textbf{TFAE}: "The following are equivalent" means all statements are true (or false)
    \end{itemize}
  \item Induction and Strong Induction
    \begin{itemize}
      \item Base case: P(n) is true for $n = n_0$ 
      \item Hypothesis: Assume P(k) is true for $k \geq n_0$
      \item Steps: Prove P(k+1) is true using P(K)
    \end{itemize}
  
\end{enumerate}

\end{multicols}

\begin{multicols}{2}
[
\section{Combinatorics}
Counting Principles and Inclusion Exclusion, Permutations, Combinations, and the Binomial Theorem, Pigeonhole Principles
]

\begin{enumerate}
    \item Addition and subtraction principles
    \item Multiplication and Division Principles
    \item Inclusion-Exclusion Principle
    \item Permutation: $P(n, k)=n!/(n - k)!$
    \item Combination: $C(n, k) = P(n, k)/k!$
    \item Binomial Theorem: $(x+y)^n=\sum^{n}_{k=0}{\binom{n}{k}x^{n-k}y^k}$
    \item Pascal's Triangle: $\binom{n + 1}{k} = \binom{n}{k - 1} + \binom{n}{k}$
    \item $\binom{n}{k} = \binom{n}{n - k}$, and $\sum^n_{k = 0}{\binom{n}{k}} = 2^n$
    \item Pigeonhole Principle: We have n + 1 items and we want to put them in n boxes. Then one box must contain at lease 2 items.
    \item Strong Pigeonhole Principle Corollary: Let $n$ and $r$ be positive integers. If $n(r - 1) + 1$ objects are distributed into $n$ boxes, then at least one of the boxes contains $r$ or more of the objects.

\end{enumerate}

\end{multicols}

\begin{multicols}{2}
[
\section{Number Theory}
Divisibility, GCD, LCM, Euclidean Algorithm, Bézout's Identity and extended GCD, CRT and FLT. Coprime or relatively prime.
]

\begin{enumerate}
    \item Let $a, b \in \mathbb{Z}$ and $b > 0$. $a$ is congruent to $b$ module $m$,
    denoted by $a \equiv b (mod\, m)$, if $mod(a, m) = mod(b, m)$
    \item Denote [0, m) by $\mathbb{Z}/m\mathbb{Z}$ or $\mathbb{Z}_m$
    \item When a number is divisible only by 1 and itself, it is a \textbf{prime} number, otherwise it is a \textbf{composite} number.
    \item $GCD(a, b) \times LCM(a, b) = a \times b$
    \item Bézout Coefficient s, t: $GCD(a, b) = s \times a + t \times b$
    \item $extGCD(a, b) = (a, 1, 0)$ if $b = 0$ \\
    or $extGCD(a, b) = (g, t, s - q \times t)$, $(g, s, t) = extGCD(b, r)$
    \item FLT: $a^p \equiv a (mod\, p)$ p is prime and $p \not \mid a$
\end{enumerate}

\end{multicols}

\end{document}
